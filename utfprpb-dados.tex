%%%% utfprpb-dados.tex, 2019/12/03
%%%% Copyright (C) 2020 Vinicius Pegorini (vinicius@utfpr.edu.br)
%%
%% This work may be distributed and/or modified under the conditions of the
%% LaTeX Project Public License, either version 1.3 of this license or (at your
%% option) any later version.
%% The latest version of this license is in
%%   http://www.latex-project.org/lppl.txt
%% and version 1.3 or later is part of all distributions of LaTeX version
%% 2005/12/01 or later.
%%
%% This work has the LPPL maintenance status `maintained'.
%%
%% The Current Maintainer of this work is Vinicius Pegorini.
%%
%% This work consists of the files utfprpb.cls, utfprpb.tex, and
%% utfprpb-dados.tex.
%%
%% A brief description of this work is in readme.txt.

%% Documento
\TipoDeDocumento{Trabalho de Conclusão de Curso de Graduação}%% Tipo de documento: "Tese", "Dissertação" ou "Trabalho de Conclusão de Curso de Graduação", "Estágio Supervisionado"
\DocumentType{Final Coursework}%% Document type: "Thesis", "Dissertation" ou "Final Coursework"
\NivelDeFormacao{Tecnólogo}%% Nível de formação: "Doutorado", "Mestrado", "Bacharelado" ou "Tecnólogo"
\FormationLevel{Technology}%% Formation level: "Doctorate", "Master's Degree", "Bachelor's Degree" ou "Technology"
\TituloDoDocumento{%% Título do documento
  Título do presente trabalho acadêmico
}
\DocumentTitle{%% Document title
  Title of this academic work
}
\TituloEmMultiplasLinhas{%% Título em multiplas linhas na capa e nas folhas de rosto e de aprovação (\par indica a quebra de linha)
  Título do presente trabalho acadêmico
}
%% Se necessário quebrar linha no título:
% Título do presente trabalho acadêmico: \par subtítulo do presente trabalho acadêmico

%% Descrição do documento na folha de rosto 
%\DescricaoDoDocumento{ \imprimirtipodedocumento\ apresentado(a) como requisito parcial à obtenção do título de \imprimirtitulopretendido\ em \imprimircurso, do \imprimirppgoudepartamento, da \imprimirinstituicao.}
%% Descrição do documento TCC1
\DescricaoDoDocumento{ \imprimirtipodedocumento,\ apresentado à disciplina de Trabalho de Conclusão de Curso 1, do Cuso Superior de \imprimircurso, do \imprimirppgoudepartamento, da \imprimirinstituicao\ -- Câmpus \imprimircidade, como requisito parcial para obtenção do título de  \imprimirtitulopretendido\ .}

%% Descrição do documento TCC2
%\DescricaoDoDocumento{ \imprimirtipodedocumento,\ apresentado(a) à disciplina de Trabalho de Conclusão de Curso 2, do Cuso Superior de \imprimircurso, do \imprimirppgoudepartamento, da \imprimirinstituicao\ -- Câmpus \imprimircidade, como requisito parcial para obtenção do título de  \imprimirtitulopretendido\ .}

%% Descrição do documento Estágio
%\DescricaoDoDocumento{ \imprimirtipodedocumento,\ apresentado à disciplina de Estágio Curricular Supervisionado, do Cuso Superior de \imprimircurso, do \imprimirppgoudepartamento, da \imprimirinstituicao\ -- Câmpus \imprimircidade.}


\TituloDaFichaCatalografica{%% Título da ficha catalográfica
  Ficha catalográfica elaborada pelo Departamento de Biblioteca da \par \imprimirinstituicao, Câmpus \imprimircidade \par n.
  \imprimirnumerodapublicacao
}
\TextoDeAprovacao{%% Texto de aprovação
  %% Exemplo de texto de aprovação para Tese ou Dissertação (descomente a próxima linha para utilizá-lo):
  Esta \imprimirtipodedocumento\ foi apresentada às 00:00 de \imprimirdia\ de \imprimirmesporextenso\ de \imprimirano\ como requisito parcial para a obtenção do título de \imprimirtitulopretendido\ em \imprimircurso, na área de concentração em \imprimirareadeconcentracao\ e na linha de pesquisa em (Nome da Linha de Pesquisa), do Programa de Pós-Graduação em \imprimircurso. O(A) candidato(a) foi arguido(a) pela Banca Examinadora composta pelos professores abaixo citados. Após deliberação, a Banca Examinadora considerou o trabalho aprovado.
  %% Exemplo de texto de aprovação para Trabalho de Conclusão de Curso (descomente a próxima linha para utilizá-lo):
  % Este \imprimirtipodedocumento\ foi apresentado em \imprimirdia\ de \imprimirmesporextenso\ de \imprimirano\ como requisito parcial para a obtenção do título de \imprimirtitulopretendido\ em \imprimircurso. O(A) candidato(a) foi arguido(a) pela Banca Examinadora composta pelos professores abaixo assinados. Após deliberação, a Banca Examinadora considerou o trabalho aprovado.
}
\AvisoDeAprovacao{%% Aviso de aprovação
  %% Exemplo de aviso de aprovação para Tese ou Dissertação (descomente a próxima linha para utilizá-lo):
  A Folha de Aprovação assinada encontra-se no \par Departamento de Registros Acadêmicos da UTFPR -- Câmpus \imprimircidade
  %% Exemplo de aviso de aprovação para Trabalho de Conclusão de Curso (descomente a próxima linha para utilizá-lo):
  % -- O Termo de Aprovação assinado encontra-se na Coordenação do Curso --
}
\NumeroDaTeseOuDissertacao{00/\imprimirano}%% Número da Tese ou Dissertação - Fornecido pelo programa de pós-graduação
\NumeroDaPublicacao{00/\imprimirano}%% Número da publicação - Fornecido pela biblioteca
\NumeroDaFichaCatalografica{A000}%% Número da ficha catalográfica - Fornecido pela biblioteca
\CDDOuCDU{CDD 000.00}%% Classificação Decimal Dewey (CDD) ou Classificação Decimal Universal (CDU) - Fornecida pela biblioteca

%% Autor(a)
\TituloPretendido{Tecnólogo(a)}%% Título pretendido: "Doutor(a)", "Mestre(a)", "Bacharel(a)" ou "Tecnólogo(a)"
\NomeDoAutor{Nome do(a) Autor(a)}%% Nome completo do(a) autor(a)
\SobrenomeDoAutor{Último Nome}%% Último nome do(a) autor(a)
\PrenomeDoAutor{Resto do Nome do(a) Autor(a)}%% Nome do(a) autor(a) sem último nome

%% Outro(a) autor(a) (Trabalho de Conclusão de Curso)
\AtribuiOutroAutor{false}%% Insere ou remove outro(a) autor(a): "true" ou "false"
\NomeDoOutroAutor{Nome do(a) Outro(a) Autor(a)}%% Nome completo do(a) outro(a) autor(a)
\SobrenomeDoOutroAutor{Último Nome}%% Último nome do(a) outro(a) autor(a)
\PrenomeDoOutroAutor{Resto do Nome do(a) Outro(a) Autor(a)}%% Nome do(a) outro(a) autor(a) sem último nome

%% Orientador(a)
\AtribuicaoOrientador{Orientador(a)}%% Atribuição "Orientador(a)"
\TituloDoOrientador{Prof(a). Dr(a).}%% Título do(a) orientador(a)
\NomeDoOrientador{Nome do(a) Orientador(a)}%% Nome completo do(a) orientador(a)
\SobrenomeDoOrienador{Último Nome}%% Último nome do(a) orientador(a)
\PrenomeDoOrientador{Resto do Nome do(a) Orientador(a)}%% Nome do(a) orientador(a) sem último nome

%% Coorientador(a)
\AtribuiCoorientador{false}%% Insere ou remove o(a) coorientador(a): "true" ou "false"
\AtribuicaoCoorientador{Coorientador(a)}%% Atribuição "Coorientador(a)"
\TituloDoCoorientador{Prof(a). Dr(a).}%% Título do(a) coorientador(a)
\NomeDoCoorientador{Nome do(a) Coorientador(a)}%% Nome completo do(a) coorientador(a)
\SobrenomeDoCoorienador{Último Nome}%% Último nome do(a) coorientador(a)
\PrenomeDoCoorientador{Resto do Nome do(a) Coorientador(a)}%% Nome do(a) coorientador(a) sem último nome

%% Instituição
\Instituicao{Universidade Tecnológica Federal do Paraná}%% Nome da instituição
%\Institution{Federal University of Technology -- Paraná}%% Institution name
\Institution{Universidade Tecnológica Federal do Paraná}%% Institution name
\Cidade{Pato Branco}%% Nome da cidade (câmpus)
\Diretoria{Graduação e Educação Profissional}%% Diretoria: "Graduação e Educação Profissional" ou "Pesquisa e Pós-Graduação"
\Departamento{Departamento Acadêmico de Informática}%% Nome do departamento ou da coordenação
\Curso{Tecnologia em Análise e Desenvolvimento de Sistemas}%% Nome do curso
\Course{Analysis and Systems Development Technology}%% Course name
\AreaDeConcentracao{Nome da Área de Concentração}%% Nome da área de concentração
\TituloDoResponsavelTCC{Prof(a). Dr(a).}%% Título do(a) responsável pelos TCC
\NomeDoResponsavelTCC{Nome do(a) Responsável}%% Nome completo do(a) responsável pelos TCC
\AtribuicaoCoordenador{Coordenador(a)}%% Atribuição "Coordenador(a)" do curso
\TituloDoCoordenador{Prof(a). Dr(a).}%% Título do(a) coordenador(a) do curso
\NomeDoCoordenador{Nome do(a) Coordenador(a)}%% Nome completo do(a) coordenador(a) do curso
\SiglaDoPPG{PPG}%% Sigla do programa de pós-graduação: PPGBIOTEC, PPGCC, PPGECT, PPGEE, PPGEM, PPGEP ou PPGEQ

%% Banca examinadora: 3 membros (Trabalho de Conclusão de Curso ou Dissertação); 5 a 7 membros (Tese)
\MembroAIgualOrientador{true}%% Insere ou remove o membro A igual ao(à) orientador(a): "true" ou "false"
\MembroA{Nome do Membro A}%% Nome completo do membro A - Presidente (automático se orientador(a))
\TituloDoMembroA{Prof(a). Dr(a).}%% Título do membro A - Presidente (automático se orientador(a))
\InstituicaoDoMembroA{Instituição do Membro A}%% Nome da instituição do membro A - Presidente (automático se orientador(a))
\MembroB{Nome do Membro B}%% Nome completo do membro B
\TituloDoMembroB{Prof(a). Dr(a).}%% Título do membro B
\InstituicaoDoMembroB{Instituição do Membro B}%% Nome da instituição do membro B
\MembroC{Nome do Membro C}%% Nome completo do membro C
\TituloDoMembroC{Prof(a). Dr(a).}%% Título do membro C
\InstituicaoDoMembroC{Instituição do Membro C}%% Nome da instituição do membro C
\MembroD{Nome do Membro D}%% Nome completo do membro D
\TituloDoMembroD{Prof(a). Dr(a).}%% Título do membro D
\InstituicaoDoMembroD{Instituição do Membro D}%% Nome da instituição do membro D
\MembroE{Nome do Membro E}%% Nome completo do membro E
\TituloDoMembroE{Prof(a). Dr(a).}%% Título do membro E
\InstituicaoDoMembroE{Instituição do Membro E}%% Nome da instituição do membro E
\AtribuiMembroF{false}%% Insere ou remove o Membro F: "true" ou "false"
\MembroF{Nome do Membro F}%% Nome completo do membro F
\TituloDoMembroF{Prof(a). Dr(a).}%% Título do membro F
\InstituicaoDoMembroF{Instituição do Membro F}%% Nome da instituição do membro F
\AtribuiMembroG{false}%% Insere ou remove o Membro G: "true" ou "false"
\MembroG{Nome do Membro G}%% Nome completo do membro G
\TituloDoMembroG{Prof(a). Dr(a).}%% Título do membro G
\InstituicaoDoMembroG{Instituição do Membro G}%% Nome da instituição do membro G

%% Data da defesa
\Dia{1}%% Dia
\MesPorExtenso{janeiro}%% Mês por extenso
\Ano{2020}%% Ano

%% Palavras-chave e keywords
\NumeroDePalavrasChave{5}%% Número de palavras-chave (máximo 5)
\PalavraChaveA{Palavra-chave A}%% Palavra-chave A
\PalavraChaveB{Palavra-chave B}%% Palavra-chave B
\PalavraChaveC{Palavra-chave C}%% Palavra-chave C
\PalavraChaveD{Palavra-chave D}%% Palavra-chave D
\PalavraChaveE{Palavra-chave E}%% Palavra-chave E
\NumeroDeKeywords{5}%% Número de keywords (máximo 5)
\KeywordA{Keyword A}%% Keyword A
\KeywordB{Keyword B}%% Keyword B
\KeywordC{Keyword C}%% Keyword C
\KeywordD{Keyword D}%% Keyword D
\KeywordE{Keyword E}%% Keyword E
