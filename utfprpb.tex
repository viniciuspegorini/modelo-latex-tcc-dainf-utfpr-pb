%%%% utfprpb.tex, 2019/12/03
%%%% Copyright (C) 2020 Vinicius Pegorini (vinicius@utfpr.edu.br)
%%
%% This work may be distributed and/or modified under the conditions of the
%% LaTeX Project Public License, either version 1.3 of this license or (at your
%% option) any later version.
%% The latest version of this license is in
%%   http://www.latex-project.org/lppl.txt
%% and version 1.3 or later is part of all distributions of LaTeX version
%% 2005/12/01 or later.
%%
%% This work has the LPPL maintenance status `maintained'.
%%
%% The Current Maintainer of this work is Vinicius Pegorini.
%%
%% This work consists of the files utfprpb.cls, utfprpb.tex, and
%% utfprpb-dados.tex.
%%
%% A brief description of this work is in readme.md.

%% Classe e opções de documento
\documentclass[%% Opções
%% -- Opções da classe memoir --
  12pt,%% Tamanho da fonte: 10pt, 11pt, 12pt, etc.
  a4paper,%% Tamanho do papel: a4paper (A4), letterpaper (carta), etc.
  % fleqn,%% Alinhamento das equações à esquerda (comente para alinhamento centralizado)
  % leqno,%% Numeração das equações no lado esquerdo (comente para lado direito)
  oneside,%% Impressão dos elementos textuais e pós-textuais: oneside (anverso) ou twoside (anverso e verso, se mais de 100 p.)
  openright,%% Impressão da primeira página dos capítulos: openright (anverso), openleft (verso) ou openany (anverso e verso)
%% -- Opções da classe abntex2 --
  sumario = abnt-6027-2012,%% Formatação do sumário: tradicional (estilo tradicional) ou abnt-6027-2012 (norma ABNT 6027-2012)
  chapter = TITLE,%% Títulos de capítulos em maiúsculas (comente para desabilitar)
  section = TITLE,%% Títulos de seções secundárias em maiúsculas (comente para desabilitar)
  % subsection = TITLE,%% Títulos de seções terciárias em maiúsculas (comente para desabilitar)
  % subsubsection = TITLE,%% Títulos de seções quartenárias em maiúsculas (comente para desabilitar)
%% -- Opções da classe utfprpgtex --
  pretextualoneside,%% Impressão dos elementos pré-textuais: pretextualoneside (anverso) ou pretextualtwoside (anverso e verso)
  fontetimes,%% Fonte do texto: fontetimes (times), fontearial (arial) ou fontecourier (courier)
  % vinculoscoloridos,%% Cores nos vínculos (citações, arquivos, links, url, etc.) (comente para desabilitar)
  semrecuonosumario,%% Remoção do recuo dos itens no sumário (comente para adição do recuo, se estilo tradicional)
  usemakeindex,%% Compilação de glossários e índices utilizando makeindex (comente para desabilitar)
  % legendascentralizadas,%% Alinhamento das legendas centralizado (comente para alinhamento à esquerda)
  %aprovacaoestiloppg,%% Folha de aprovação do programa de pós-graduação no estilo do PPG (comente para estilo padrão)
  pardeassinaturas,%% Assinaturas na folha de aprovação em até duas colunas (comente para em uma única coluna)
  % linhasdeassinaturas,%% Linhas de assinaturas na folha de aprovação (comente para remover as linhas)
%% -- Opções do pacote babel --
  english,%% Idioma adicional para hifenização
  french,%% Idioma adicional para hifenização
  spanish,%% Idioma adicional para hifenização
  brazil,%% Idioma principal do documento (último da lista)
]{utfprpb}%% Classe utfprpb

%% Pacotes carregados nas classes:
%%   memoir: abstract, appendix, array, booktabs, ccaption, chngcntr, chngpage, dcolumn, delarray, enumerate, epigraph, framed,
%%           ifmtarg, ifpdf, index, makeidx, moreverb, needspace, newfile, nextpage, parskip, patchcmd, setspace, shortvrb, showidx,
%%           tabularx, titleref, titling, tocbibind, tocloft, verbatim, verse.
%%   memoir (similares): crop, fancyhdr, geometry, sidecap, subfigure, titlesec.
%%   abntex2: babel, bookmark, calc, enumitem, ifthen, hyperref, textcase.
%%   utfprpgtex: abntex2cite, ae, algorithmic, amsmath, backref, breakurl, caption, cmap, color, eepic, epic, epsfig, etoolbox,
%%               fancyhdr, fix-cm, fontenc, glossaries, graphics, graphicx, helvet, hyphenat, indentfirst, inputenc, lastpage,
%%               morewrites, nomencl, sfmath, sistyle, substr, times, xtab.

%% Pacotes adicionais (\usepackage[options]{package})
\usepackage{bigdelim, booktabs, colortbl, longtable, multirow}%% Ferramentas para tabelas
\usepackage{amssymb, amstext, amsthm, icomma}%% Ferramentas para linguagem matemática
\usepackage{pifont, textcomp, wasysym}%% Símbolos de texto
\usepackage{lipsum}				% para geração de dummy text
\usepackage{subfig}             % para adicionar figuras lado a lado no texto                    
\usepackage{pdfpages}           % para adicionar documentos pdf ao trabalho

%% Comandos personalizados (\newcommand{name}[num]{definition})
\newcommand{\cpp}{\texttt{C$++$}}%% C++
\newcommand{\latex}{\LaTeX}%% LaTeX
\newcommand{\ds}{\displaystyle}%% Tamanho normal das equações
\newcommand{\bsym}[1]{\boldsymbol{#1}}%% Texto no modo matemático em negrito
\newcommand{\mr}[1]{\mathrm{#1}}%% Texto no modo matemático normal (não itálico)
\newcommand{\der}{\mr{d}}%% Operador diferencial
\newcommand{\deri}[2]{\frac{\der #1}{\der #2}}%% Derivada ordinária
\newcommand{\derip}[2]{\frac{\partial #1}{\partial #2}}%% Derivada parcial
\newcommand{\pare}[1]{\left( #1 \right)}%% Parênteses
\newcommand{\colc}[1]{\left[ #1 \right]}%% Colchetes
\newcommand{\chav}[1]{\left\lbrace #1 \right\rbrace}%% Chaves
\newcommand{\sen}{\operatorname{sen}}%% Operador seno
\newcommand{\senh}{\operatorname{senh}}%% Operador seno hiperbólico
\newcommand{\tg}{\operatorname{tg}}%% Operador tangente
\newcommand{\tgh}{\operatorname{tgh}}%% Operador tangente hiperbólico
\newcommand{\seqref}[1]{Equação~\eqref{#1}}%% Referência de uma única equação
\newcommand{\meqref}[1]{Equações~\eqref{#1}}%% Referência de múltiplas equações
\newcommand{\citep}[1]{\cite{#1}}%% Atalho para citação implícita
\newcommand{\citet}[1]{\citeonline{#1}}%% Atalho para citação explícita
\newcommand{\citepa}[1]{(\citeauthor{#1})}%% Atalho para citação implícita (somente autor)
\newcommand{\citeta}[1]{\citeauthoronline{#1}}%% Atalho para citação explícita (somente autor)
\newcommand{\citepy}[1]{(\citeyear{#1})}%% Atalho para citação implícita (somente ano)
\newcommand{\citety}[1]{\citeyear{#1}}%% Atalho para citação explícita (somente ano)

%% Arquivo de dados do modelo de documento LaTeX para produção de trabalhos acadêmicos da UTFPR
\input{./utfprpb-dados}%% Realize as modificações pertinentes no arquivo "utfprpb-dados.tex"

%% Ferramenta para criação de índices
\makeindex%% Não comente esta linha

%% Ferramenta para criação de glossários
\makeglossaries%% Não comente esta linha
\include{./PreTexto/entradas-siglas}%% Entradas da lista de abreviaturas e siglas - Comente para remover este item
\include{./PosTexto/entradas-glossario}%% Entradas do glossário - Comente para remover este item

%% Ferramenta para criação de nomenclaturas
\makenomenclature%% Não comente esta linha

%% Início do documento
\begin{document}%% Não comente esta linha

%% Formatação de páginas de elementos pré-textuais
\pretextual%% Não comente esta linha

%% Capa
\incluircapa%% Comente para remover este item

%% Folha de rosto (* coloca a ficha bibliográfica no verso)
\incluirfolhaderosto*%% Comente para remover este item

%% Ficha catalográfica (teses e dissertações)
%\incluirfichacatalografica%% Comente para remover este item

%% Errata
%\include{./PreTexto/errata}%% Comente para remover este item

%% Folha de aprovação
%\incluirfolhadeaprovacao%% Para adicionar no formato de texto
%\includepdf[scale=1.0,pages=1]{./PreTexto/folha-aprovacao.pdf} % para adicionar o pdf enviado pelo professor apenas substitua o documento folha-aprovacao.pdf dentro da pasta PreTexto

%% Dedicatória
%%%%% DEDICATÓRIA
%%
%% Texto em que o autor presta homenagem ou dedica seu trabalho.

\begin{dedicatoria}%% Ambiente dedicatoria
Dedicatória: Oferecimento do trabalho à determinada pessoa ou pessoas, instituição. É opcional.
\end{dedicatoria}
%% Comente para remover este item

%% Agradecimentos
%%%%% AGRADECIMENTOS
%%
%% Texto em que o autor faz agradecimentos dirigidos àqueles que contribuíram de maneira relevante à elaboração do trabalho.

\begin{agradecimentos}%% Ambiente agradecimentos
Folha que contém manifestação de reconhecimento a pessoas e/ou instituições que realmente contribuíram com o autor, devendo ser expressos de maneira simples.

A página de agradecimento assim com a página de dedicatória e a próxima página que representa uma epígrafe são opcionais.

Na epígrafe o autor usa uma citação, seguida de indicação de autoria e ano, relacionada com a matéria tratada no corpo do trabalho.
\end{agradecimentos}
%% Comente para remover este item

%% Epígrafe
%\include{./PreTexto/epigrafe}%% Comente para remover este item

%% Resumo
%%%% RESUMO
%%
%% Apresentação concisa dos pontos relevantes de um texto, fornecendo uma visão rápida e clara do conteúdo e das conclusões do
%% trabalho.

\begin{resumoutfpr}%% Ambiente resumoutfpr
O resumo deve fornecer uma visão geral do trabalho, ainda que bem sucinta, incluindo o assunto, o objetivo principal, a justificativa do trabalho (do assunto e/ou das tecnologias utilizadas), como o mesmo foi desenvolvido e uma visão breve dos resultados e da conclusão. Esses itens devem formar um texto e não ser uma sequência de tópicos. Não deve conter citações. Não deve ultrapassar 500 palavras. É escrito em um único parágrafo e sem espaço de tabulação no início.
\end{resumoutfpr}
%% Comente para remover este item

%% Abstract
%%%% ABSTRACT
%%
%% Versão do resumo para idioma de divulgação internacional.

\begin{abstractutfpr}%% Ambiente abstractutfpr
Tradução do resumo para a língua inglesa.
\end{abstractutfpr}
%% Comente para remover este item

%% Lista de algoritmos
%\incluirlistadealgoritmos%% Comente para remover este item

%% Lista de ilustrações
\incluirlistadeilustracoes%% Comente para remover este item

%% Lista de Fotografias
\incluirlistadefotografias %% Comente para remover este item

%% Lista de Gráficos
\incluirlistadegraficos %% Comente para remover este item


%% Lista de tabelas
\incluirlistadetabelas%% Comente para remover este item

%% Lista de quadros
\incluirlistadequadros

%% Listagem de códigos fonte
\incluirlistadecodigosfonte

%% Lista de abreviaturas, siglas e acrônimos
\incluirlistadeacronimos{glossaries}%% Opções: "glossaries" (pacote) ou "file" (arquivo) ou "none" (desabilita)

%% Lista de símbolos
\incluirlistadesimbolos{nomencl}%% Opções: "nomencl" (pacote) ou "file" (arquivo) ou "none" (desabilita)

%% Sumário
\incluirsumario%% Comente para remover este item

%% Formatação de páginas de elementos textuais
\textual%% Não comente esta linha

%% Parte
% \part{Introdução}%% Comente para remover este item

%% Capítulo
\include{./Capitulo1/capitulo1}%% Comente para remover este item

%% Parte
% \part{Desenvolvimento}%% Comente para remover este item

%% Capítulo
\include{./Capitulo2/capitulo2}%% Comente para remover este item

%% Capítulo
\include{./Capitulo3/capitulo3}%% Comente para remover este item

%% Capítulo
\include{./Capitulo4/capitulo4}%% Comente para remover este item

%% Parte
% \part{Conclusão}%% Comente para remover este item

%% Capítulo
\include{./Capitulo5/capitulo5}%% Comente para remover este item

%% Capítulos após este comando criam marcadores do pdf na raiz
% \phantompart%% Comente para remover este item

%% Capítulo 6 - esse capítulo contém exemplos para melhor uso do modelo Latex
%% Na versão final do TCC esse capítulo deve ser removido utilizando o sinal %
\include{./Capitulo6-Exemplo/capitulo6-exemplo}%% Comente para remover este item

%% Formatação de páginas de elementos pós-textuais
\postextual%% Não comente esta linha

%% Arquivos de referências
\arquivosdereferencias{%% Arquivos bibtex sem a extensão .bib e separados por vírgula - Não comente esta linha
  ./PosTexto/exemplos-referencias,%% Arquivo de referências - Comente para remover este item
  ./PosTexto/referencias%% Arquivo de referências - Comente para remover este item
}%% Não comente esta linha

%% Glossário
%\incluirglossario %% Comente para remover este item

%% Arquivos de apêndices
\begin{arquivosdeapendices}%% Os arquivos de apêndices devem se incluídos neste ambiente - Não comente esta linha
  %\partapendices%% Página de início dos apêndices - adiciona uma página com o título Apêndices
  %% Capítulo de exemplo
  \include{./PosTexto/apendicea}%% Apêndice - Comente para remover este item
  \include{./PosTexto/apendiceb}%% Apêndice - Comente para remover este item
\end{arquivosdeapendices}%% Não comente esta linha

%% Arquivos de anexos
\begin{arquivosdeanexos}%% Os arquivos de anexos devem se incluídos neste ambiente - Não comente esta linha
  %\partanexos%% Página de início dos anexos - adiciona uma página com o título Anexos

  \include{./PosTexto/anexoa}%% Anexo - Comente para remover este item
  \include{./PosTexto/anexob}%% Anexo - Comente para remover este item
\end{arquivosdeanexos}%% Não comente esta linha

%% Índice - Adiciona um índice remissivo.
%\incluirindice%% Comente para remover este item

%% Fim do documento
\end{document}%% Não comente esta linha
